%
% Modified by Megan Patnott
% Last Change: Jan 18, 2013
%
%%%%%%%%%%%%%%%%%%%%%%%%%%%%%%%%%%%%%%%%%%%%%%%%%%%%%%%%%%%%%%%%%%%%%%%%
%
% Modified by Sameer Vijay
% Last Change: Tue Jul 26 2005 13:00 CEST
%
%%%%%%%%%%%%%%%%%%%%%%%%%%%%%%%%%%%%%%%%%%%%%%%%%%%%%%%%%%%%%%%%%%%%%%%%
%
% Sample Notre Dame Thesis/Dissertation
% Using Donald Peterson's ndthesis classfile
%
% Written by Jeff Squyres and Don Peterson
%
% Provided by the Information Technology Committee of
%   the Graduate Student Union
%   http://www.gsu.nd.edu/
%
% Nothing in this document is serious except the format.  :-)
%
% If you have any suggestions, comments, questions, please send e-mail
% to: ndthesis@gsu.nd.edu
%
%%%%%%%%%%%%%%%%%%%%%%%%%%%%%%%%%%%%%%%%%%%%%%%%%%%%%%%%%%%%%%%%%%%%%%%%


%
% Chapter 5
%

\chapter{CONCLUSIONS}
In this dissertation, I have presented the three newly developed real-space methods: Shifted Potential (SP), Gradient Shifted Force (GSF), and Taylor Shifted Force (TSF), for evaluating electrostatic interactions between multipoles in molecular simulations. I have also computed various static and dynamic properties using newly developed real space methods and compared result with the Ewald sum. I have also studied the conservation of total energy using real-space as well as Ewald method. Finally, I have discussed the perturbation, fluctuation, and PMF methods for calculating dielectric properties for dipolar and quadrupolar systems.

The energy constant evaluated for the dipolar and quadrupolar crystal using SP and GSF methods show excellent agreement with the analytical result. The TSF method performs poorly in predicting the energy constant for both dipolar and quadrupolar crystal. We need to take larger cutoff radius and suitable damping alpha to predict energy constant using TSF method. For the dipolar crystal, higher damping favors the quick convergence of the energy constant when plotted against cutoff radius. But the higher damping can presents additional issues while calculating energy constant for quadrupolar crystal, hence the small damping parameter is suitable for the quadrupolar system.     

The SP method is the mutipolar version of the Wolf's shifted potential for the charge-charge interaction, whereas the GSF and TSF method are multipolar generalization of the Damped Shifted Force (DSF) method. In the SP method, the potential energy smoothly approaches to zero at the cutoff radius but the forces and torques derived from the energies are not continuous at the cutoff radius. On the other hand, the forces and torques as well as energies evaluated in the GSF and TSF method vanishes at the cutoff radius. Since the energy evaluated from the SP method shows excellent agreement with the Ewald, it is suitable for the Monte Carlo (MC) simulation. The GSF method performs well in conserving total energy in MD simulations and produces good quantitative agreement with the Ewald energies, forces and torques, therefore better choice for MD simulations. Both SP and GSF models perform remarkably well with moderate damping parameter for reproducing static as well as dynamic properties in the liquid systems. All of these real-space methods scales \textit{linearly} with system size and are easily parallelizable, therefore reduce computational cost substantially while performing large simulations.         

The test of the dielectric properties is very important to validate the electrostatic interaction methods. I presented three methods: perturbation, fluctuation, and potential of mean force (PMF), to evaluate dielectric properties for dipolar and quadrupolar fluids. The polarizability obtained from simulations, using perturbation and fluctuation methods, shows excellent agreement with each other for all SP, GSF, and TSF methods. To calculate the dielectric constant and susceptibility from the dipolar polarization (measured from simulations), we need to take account of correction factor listed in the Chapter 4. The correction factor for TSF method is 1, which essentially signifies no correction. Therefore the dielectric constant evaluated using conducting boundary condition can produce good result for all values of damping alpha. In the other words, the susceptibility and polarizability are the same in the case of dipolar TSF method. The correction factor for SP and GSF methods depend on damping parameter as well as cutoff radius. The formula for deriving dielectric constant from polarization is very sensitive when the damping parameter is very small. Therefore moderate damping alpha between ($0.25 - 0.3~\AA^{-2}$) are always suitable choice for calculating dielectric properties using SP and GSF method.

For the quadrupolar fluid, the correction factor essentially small for SP and GSF method whereas the correction factor is comparatively large for the TSF. Different than dipolar case, the quadrupolar susceptibility but not dielectric screening measures the actual bulk properties of the quadrupolar fluid. On other hand, the dielectric constant is geometry dependent and can be derived from the susceptibility and geometry of the applied field. 

The dipolar dielectric screening calculated from the perturbation and fluctuation also agree with the screening factor measured from the PMF method for the damping parameter $0.25 - 0.3~\AA^{-2}$. Similarly the quadrupolar dielectric screening calculated between two point charges using susceptibility and geometrical factor shows good agreement with the direct screening measured using PMF method.

All of these newly developed real space methods are computationally very efficient as compared to Ewald method. The physical properties predicted by the real-space methods, with the suitable choice of damping parameter, are as accurate as the computationally expensive Ewald method. We can even predict the dielectric properties correctly using newly developed methods with the use of the correction factor. Therefore our newly developed real space methods are the suitable choice for calculating electrostatic interactions between molecules in the large molecular simulations.         
    

  



% % uncomment the following lines,
% if using chapter-wise bibliography
%
% \bibliographystyle{ndnatbib}
% \bibliography{example}