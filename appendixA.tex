%
% Modified by Sameer Vijay
% Last Change: Wed Jul 27 2005 13:00 CEST
%
%%%%%%%%%%%%%%%%%%%%%%%%%%%%%%%%%%%%%%%%%%%%%%%%%%%%%%%%%%%%%%%%%%%%%%%%
%
% Sample Notre Dame Thesis/Dissertation
% Using Donald Peterson's ndthesis classfile
%
% Written by Jeff Squyres and Don Peterson
%
% Provided by the Information Technology Committee of
%   the Graduate Student Union
%   http://www.gsu.nd.edu/
%
% Nothing in this document is serious except the format.  :-)
%
% If you have any suggestions, comments, questions, please send e-mail
% to: ndthesis@gsu.nd.edu
%
%%%%%%%%%%%%%%%%%%%%%%%%%%%%%%%%%%%%%%%%%%%%%%%%%%%%%%%%%%%%%%%%%%%%%%%%

%%%%%%%%%%%%%%%%%%%%%%%%%%%%%%%%%%%%%%%%%%%%%%%%%%%%%%%%%%%%%%%%%%%%%%%%
%
% Appendix
%
%%%%%%%%%%%%%%%%%%%%%%%%%%%%%%%%%%%%%%%%%%%%%%%%%%%%%%%%%%%%%%%%%%%%%%%%

\chapter{RADIAL FUNCTIONS FOR REAL-SPACE ELECTROSTATIC METHODS}

\section{Smith's $B_l(r)$ functions for damped-charge distributions}
\label{SmithFunc}
The following summarizes Smith's $B_l(r)$ functions and includes
formulas given in his appendix.\cite{Smith98} The first function
$B_0(r)$ is defined by
%
\begin{equation}
B_0(r)=\frac{\text{erfc}(\alpha r)}{r} = \frac{2}{\sqrt{\pi}r}
\int_{\alpha r}^{\infty} \text{e}^{-s^2} ds .
\end{equation}
%
The first derivative of this function is
%
\begin{equation}
\frac{dB_0(r)}{dr}=-\frac{1}{r^2}\text{erfc}(\alpha r)
-\frac{2\alpha}{r\sqrt{\pi}}\text{e}^{-{\alpha}^2r^2}
\end{equation}
%
which can be used to define a function $B_1(r)$:
%
\begin{equation}
B_1(r)=-\frac{1}{r}\frac{dB_0(r)}{dr}
\end{equation}
%
In general, the recurrence relation,
\begin{equation}
B_l(r)=-\frac{1}{r}\frac{dB_{l-1}(r)}{dr} 
= \frac{1}{r^2} \left[ (2l-1)B_{l-1}(r) + \frac {(2\alpha^2)^l}{\alpha \sqrt{\pi}}
\text{e}^{-{\alpha}^2r^2}
\right] ,
\end{equation}
is very useful for building up higher derivatives. As noted by Smith,
it is possible to approximate the $B_l(r)$ functions,
%
\begin{equation}
B_l(r)=\frac{(2l)!}{l!2^lr^{2l+1}} - \frac {(2\alpha^2)^{l+1}}{(2l+1)\alpha \sqrt{\pi}}
+\text{O}(r) .
\end{equation}

\newpage
\section{The $r$-dependent factors for TSF electrostatics}
\label{radialTSF}

Using the shifted damped functions $f_n(r)$ defined by:
%
\begin{equation}
f_n(r)= B_0(r) -\sum_{m=0}^{n+1} \frac {(r-r_c)^m}{m!} B_0^{(m)}(r_c)  ,
\end{equation}
%
where the superscript $(m)$ denotes the $m^\mathrm{th}$ derivative. In
this Appendix, we provide formulas for successive derivatives of this
function.  (If there is no damping, then $B_0(r)$ is replaced by
$1/r$.)  First, we find:
%
\begin{equation}
\frac{\partial f_n}{\partial r_\alpha}=\hat{r}_\alpha \frac{d f_n}{d r} .
\end{equation}
%
This formula clearly brings in derivatives of Smith's $B_0(r)$
function, and we define higher-order derivatives as follows:
%
\begin{align}
g_n(r)=& \frac{d f_n}{d r} =
B_0^{(1)}(r) -\sum_{m=0}^{n} \frac {(r-r_c)^m}{m!} B_0^{(m+1)}(r_c) \\
h_n(r)=& \frac{d^2f_n}{d r^2} =
B_0^{(2)}(r) -\sum_{m=0}^{n-1} \frac {(r-r_c)^m}{m!} B_0^{(m+2)}(r_c) \\
s_n(r)=& \frac{d^3f_n}{d r^3} =
B_0^{(3)}(r) -\sum_{m=0}^{n-2} \frac {(r-r_c)^m}{m!} B_0^{(m+3)}(r_c) \\
t_n(r)=& \frac{d^4f_n}{d r^4} =
B_0^{(4)}(r) -\sum_{m=0}^{n-3} \frac {(r-r_c)^m}{m!} B_0^{(m+4)}(r_c) \\
u_n(r)=& \frac{d^5f_n}{d r^5} =
B_0^{(5)}(r) -\sum_{m=0}^{n-4} \frac {(r-r_c)^m}{m!} B_0^{(m+5)}(r_c)  .
\end{align}
%
We note that the last function needed (for quadrupole-quadrupole interactions) is
%
\begin{equation}
u_4(r)=B_0^{(5)}(r) - B_0^{(5)}(r_c) .
\end{equation}
% The functions
% needed are listed schematically below:
% %
% \begin{eqnarray}
% f_0 \quad f_1 \qquad \qquad \quad & \nonumber \\
% g_0 \quad g_1 \quad g_2 \quad g_3 \quad &g_4 \nonumber \\
% h_1 \quad h_2 \quad h_3 \quad &h_4 \nonumber \\
% s_2 \quad s_3 \quad &s_4 \nonumber \\
% t_3 \quad &t_4 \nonumber \\
% &u_4 \nonumber .
% \end{eqnarray}
The functions $f_n(r)$ to $u_n(r)$ can be computed recursively and
stored on a grid for values of $r$ from $0$ to $r_c$.  Using these
functions, we find
%
\begin{align}
\frac{\partial f_n}{\partial r_\alpha} =&r_\alpha \frac {g_n}{r} \label{eq:b9}\\
\frac{\partial^2 f_n}{\partial r_\alpha \partial r_\beta} =&\delta_{\alpha \beta}\frac {g_n}{r} 
+r_\alpha r_\beta \left( -\frac{g_n}{r^3} +\frac{h_n}{r^2}\right) \\
\frac{\partial^3 f_n}{\partial r_\alpha \partial r_\beta \partial r_\gamma} =&
\left( \delta_{\alpha \beta} r_\gamma + \delta_{\alpha \gamma} r_\beta + 
\delta_{ \beta \gamma} r_\alpha \right)  
\left(  -\frac{g_n}{r^3} +\frac{h_n}{r^2} \right) \nonumber \\
& + r_\alpha r_\beta r_\gamma 
\left(  \frac{3g_n}{r^5}-\frac{3h_n}{r^4} +\frac{s_n}{r^3} \right) \\
\frac{\partial^4 f_n}{\partial r_\alpha \partial r_\beta \partial
  r_\gamma \partial r_\delta} =& 
\left( \delta_{\alpha \beta} \delta_{\gamma \delta} 
+ \delta_{\alpha \gamma} \delta_{\beta \delta}
 +\delta_{ \beta \gamma} \delta_{\alpha \delta} \right)
\left( - \frac{g_n}{r^3} + \frac{h_n}{r^2} \right)  \nonumber \\ 
&+ \left( \delta_{\alpha \beta} r_\gamma r_\delta
+ \text{5 permutations}
\right) \left( \frac{3 g_n}{r^5} - \frac{3h_n}{r^4} + \frac{s_n}{r^3} 
\right) \nonumber \\
&+ r_\alpha r_\beta r_\gamma r_\delta
\left(  -\frac{15g_n}{r^7} + \frac{15h_n}{r^6} - \frac{6s_n}{r^5}
+ \frac{t_n}{r^4} \right)\\
\frac{\partial^5 f_n}
{\partial r_\alpha \partial r_\beta \partial r_\gamma \partial
  r_\delta \partial r_\epsilon} =& 
\left( \delta_{\alpha \beta} \delta_{\gamma \delta} r_\epsilon
+ \text{14 permutations} \right) 
\left(  \frac{3g_n}{r^5}-\frac{3h_n}{r^4} +\frac{s_n}{r^3} \right) \nonumber \\
&+ \left( \delta_{\alpha \beta} r_\gamma r_\delta r_\epsilon
+ \text{9 permutations}
\right) \left(- \frac{15g_n}{r^7}+\frac{15h_n}{r^7} -\frac{6s_n}{r^5} +\frac{t_n}{r^4} 
\right) \nonumber \\
&+ r_\alpha r_\beta r_\gamma r_\delta r_\epsilon
\left(  \frac{105g_n}{r^9} - \frac{105h_n}{r^8} + \frac{45s_n}{r^7}
- \frac{10t_n}{r^6} +\frac{u_n}{r^5} \right) \label{eq:b13}
\end{align}
%
%
%
\newpage
\section{The $r$-dependent factors for GSF electrostatics}
\label{radialGSF}

In Gradient-shifted force electrostatics, the kernel is not expanded,
and the expansion is carried out on the individual terms in the
multipole interaction energies. For damped charges, this still brings
multiple derivatives of the Smith's $B_0(r)$ function into the
algebra. To denote these terms, we generalize the notation of the
previous appendix. For either $f(r)=1/r$ (undamped) or $f(r)=B_0(r)$
(damped),
%
\begin{align}
g(r) &= \frac{df}{d r} &&                      &&=-\frac{1}{r^2}
&&\mathrm{or~~~} -rB_1(r) \\
h(r) &= \frac{dg}{d r} &&= \frac{d^2f}{d r^2} &&= \frac{2}{r^3} &&\mathrm{or~~~}-B_1(r) + r^2 B_2(r) \\
s(r) &= \frac{dh}{d r} &&= \frac{d^3f}{d r^3} &&=-\frac{6}{r^4}&&\mathrm{or~~~}3rB_2(r) - r^3 B_3(r)\\
t(r) &= \frac{ds}{d r} &&= \frac{d^4f}{d r^4} &&= \frac{24}{r^5} &&\mathrm{or~~~} 3
B_2(r) - 6r^2 B_3(r) + r^4 B_4(r) \\
u(r) &= \frac{dt}{d r} &&= \frac{d^5f}{d r^5} &&=-\frac{120}{r^6} &&\mathrm{or~~~} -15
r B_3(r) + 10 r^3B_4(r) -r^5B_5(r).
\end{align}
%
For undamped charges, Table I lists these derivatives under the Bare
Coulomb column. Equations \ref{eq:b9} to \ref{eq:b13} are still
correct for GSF electrostatics if the subscript $n$ is eliminated.

% % uncomment the following lines,
% if using chapter-wise bibliography
%
% \bibliographystyle{ndnatbib}
% \bibliography{example}