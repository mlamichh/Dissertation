%
% Modified by Sameer Vijay
% Last Change: Wed Jul 27 2005 13:00 CEST
%
%%%%%%%%%%%%%%%%%%%%%%%%%%%%%%%%%%%%%%%%%%%%%%%%%%%%%%%%%%%%%%%%%%%%%%%%
%
% Sample Notre Dame Thesis/Dissertation
% Using Donald Peterson's ndthesis classfile
%
% Written by Jeff Squyres and Don Peterson
%
% Provided by the Information Technology Committee of
%   the Graduate Student Union
%   http://www.gsu.nd.edu/
%
% Nothing in this document is serious except the format.  :-)
%
% If you have any suggestions, comments, questions, please send e-mail
% to: ndthesis@gsu.nd.edu
%
%%%%%%%%%%%%%%%%%%%%%%%%%%%%%%%%%%%%%%%%%%%%%%%%%%%%%%%%%%%%%%%%%%%%%%%%

%%%%%%%%%%%%%%%%%%%%%%%%%%%%%%%%%%%%%%%%%%%%%%%%%%%%%%%%%%%%%%%%%%%%%%%%
%
% Appendix
%
%%%%%%%%%%%%%%%%%%%%%%%%%%%%%%%%%%%%%%%%%%%%%%%%%%%%%%%%%%%%%%%%%%%%%%%%

\chapter{}

\section{Contraction of the quadrupolar tensor with the traceless
  quadrupole moment}
\label{ap:quadContraction}
For quadrupolar liquids modeled using point quadrupoles, the
interaction tensor is shown in Eq. (\ref{quadRadial}).  The Fourier
transformation of this tensor for $ \mathbf{k} = 0$ is,
\begin{equation}
\tilde{T}_{\alpha\beta\gamma\delta}(0) = \int_V T_{\alpha\beta\gamma\delta}(\mathbf{r}) d \mathbf{r}
\end{equation}
On the basis of symmetry, the 81 elements can be placed in four
different groups: $\tilde{T}_{aaaa}$, $\tilde{T}_{aaab}$,
$\tilde{T}_{aabb}$, and $\tilde{T}_{aabc}$, where $a$, $b$, and $c$,
and can take on distinct values from the set $\left\{x, y, z\right\}$.
The elements belonging to each of these groups can be obtained using
permutations of the indices.  Integration of all of the elements shows
that only the groups with indices ${aaaa}$ and ${aabb}$ are non-zero.

We can derive values of the components of $\tilde{T}_{aaaa}$ and
$\tilde{T}_{aabb}$ as follows;
\begin{eqnarray}
\tilde{T}_{xxxx}(0) &=&
\int_{\textrm{V}} 
\left[ 3v_{41}(R)+6x^2v_{42}(r)/r^2 + x^4\,v_{43}(r)/r^4 \right] d\mathbf{r} \nonumber \\ 
&=&12\pi \int_0^{r_c} 
\left[ v_{41}(r)+\frac{2}{3} v_{42}(r) + \frac{1}{15}v_{43}(r) \right] r^2\,dr =
\mathrm{12 \pi B}
\end{eqnarray}
and 
\begin{eqnarray}
  \tilde{T}_{xxyy}(0)&=&
                         \int_{\textrm{V}} 
                         \left[ v_{41}(R)+(x^2+y^2) v_{42}(r)/r^2 + x^2 y^2\,v_{43}(r)/r^4 \right] d\mathbf{r} \nonumber \\
                     &=&4\pi \int_0^{r_c}
                         \left[ v_{41}(r)+\frac{2}{3} v_{42}(r) + \frac{1}{15}v_{43}(r) \right] r^2\,dr =
                         \mathrm{4 \pi B}.
\end{eqnarray}
These integrals yield the same values for all permutations of the
indices in both tensor element groups.  In Eq.~\ref{fourierQuadZeroK}, for a particular value of the quadrupolar
polarization $\tilde{\Theta}_{aa}$ we can contract
$\tilde{T}_{aa\gamma\delta}(0)$ with $\tilde{\Theta}_{\gamma\delta}$,
using the traceless properties of the quadrupolar moment,
\begin{eqnarray}
\tilde{T}_{xx\gamma\delta}(0)\tilde{\Theta}_{\gamma\delta}(0) &=& \tilde{T}_{xxxx}(0)\tilde{\Theta}_{xx}(0) + \tilde{T}_{xxyy}(0)\tilde{\Theta}_{yy}(0) + \tilde{T}_{xxzz}(0)\tilde{\Theta}_{zz}(0) \nonumber \\
&=& 12 \pi \mathrm{B}\tilde{\Theta}_{xx}(0) +
    4 \pi \mathrm{B}\tilde{\Theta}_{yy}(0) +
    4 \pi \mathrm{B}\tilde{\Theta}_{zz}(0) \nonumber \\
&=& 8 \pi \mathrm{B}\tilde{\Theta}_{xx}(0) + 4 \pi
    \mathrm{B}\left(\tilde{\Theta}_{xx}(0)+\tilde{\Theta}_{yy}(0) +
    \tilde{\Theta}_{zz}(0)\right) \nonumber \\
&=& 8 \pi \mathrm{B}\tilde{\Theta}_{xx}(0)
\end{eqnarray} 
Similarly for a quadrupolar polarization $\tilde{\Theta}_{xy}$ in
Eq.~\ref{fourierQuadZeroK}, we can contract
$\tilde{T}_{xy\gamma\delta}(0)$ with $\tilde{\Theta}_{\gamma\delta}$,
using the only surviving terms of the tensor, 
\begin{eqnarray}
\tilde{T}_{xy\gamma\delta}(0)\tilde{\Theta}_{\gamma\delta}(0) &=& \tilde{T}_{xyxy}(0)\tilde{\Theta}_{xy}(0) + \tilde{T}_{xyyx}(0)\tilde{\Theta}_{yx}(0) \nonumber \\
&=& 4 \pi \mathrm{B}\tilde{\Theta}_{xy}(0) +
    4 \pi \mathrm{B}\tilde{\Theta}_{yx}(0) \nonumber \\
&=& 8 \pi \mathrm{B}\tilde{\Theta}_{xy}(0)
\end{eqnarray}
Here, we have used the symmetry of the quadrupole tensor to combine
the symmetric terms. Therefore we can write matrix contraction for
$\tilde{T}_{\alpha\beta\gamma\delta}(\mathrm{0})$ and
$ \tilde{\Theta}_{\gamma\delta}(\mathrm{0})$ in a general form,
\begin{equation}
\tilde{T}_{\alpha\beta\gamma\delta}(\mathrm{0})\tilde{\Theta}_{\gamma\delta}(\mathrm{0})
= 8 \pi \mathrm{B} \tilde{\Theta}_{\alpha\beta}(\mathrm{0}),
\label{contract}
\end{equation}
which is the same as Eq. (\ref{quadContraction}).

When the molecular quadrupoles are represented by point charges, the
symmetry of the quadrupolar tensor is same as for point quadrupoles
(see Eqs.~\ref{quadCharge} and (\ref{quadRadial}). However, for
molecular quadrupoles represented by point dipoles, the symmetry of
the quadrupolar tensor must be handled separately (compare
Eqs.~(\ref{quadDip}) and (\ref{quadRadial}). Although there is a
difference in symmetry, the final result (Eq.~\ref{contract}) also holds
true for dipolar representations.

\section{Quadrupolar correction factor for the Ewald-Kornfeld (EK)
  method}
The interaction tensor between two point quadrupoles in the Ewald
method may be expressed,\cite{Smith98,NeumannII83}
\begin{align}
{T}_{\alpha\beta\gamma\delta}(\mathbf{r}) = &\frac{4\pi}{V
                                              }\sum_{k\neq0}^{\infty}
                                              e^{-k^2 / 4
                                              \kappa^2} e^{-i\mathbf{k}\cdot
                                              \mathbf{r}} \left(\frac{r_\alpha r_\beta k_\delta k_\gamma}{k^2}\right)  \nonumber \\
&+ \left(\delta_{\alpha\beta}\delta_{\gamma\delta}+\delta_{\alpha\gamma}\delta_{\beta\delta}+\delta_{\alpha\delta}\delta_{\beta\gamma}\right) 
B_2(r) \nonumber \\
&- \left(\delta_{\gamma\delta} r_\alpha r_\beta +  \mathrm{ 5\; permutations}\right) B_3(r) \nonumber \\
&+ \left(r_\alpha r_\beta r_\gamma r_\delta \right)  B_4(r)
\label{ewaldTensor}
\end{align}
where $B_n(r)$ are radial functions defined in Ref.~\cite{Smith98},
\begin{align}
B_2(r)  =& \frac{3}{r^5} \left(\frac{2r\kappa e^{-r^2 \kappa^2}}{\sqrt{\pi}}+\frac{4r^3\kappa^3 e^{-r^2 \kappa^2}}{3\sqrt{\pi}}+\mathrm{erfc(\kappa r)} \right) \\
B_3(r) =& - \frac{15}{r^7}\left(\frac{2r\kappa e^{-r^2 \kappa^2}}{\sqrt{\pi}}+\frac{4r^3\kappa^3 e^{-r^2 \kappa^2}}{3\sqrt{\pi}}+\frac{8r^5\kappa^5 e^{-r^2 \kappa^2}}{15\sqrt{\pi}}+\mathrm{erfc(\kappa r)} \right) \\
B_4(r) =& \frac{105}{r^9}\left(\frac{2r\kappa e^{-r^2
          \kappa^2}}{\sqrt{\pi}}+\frac{4r^3\kappa^3 e^{-r^2 \kappa^2}}{3\sqrt{\pi}} +\frac{8r^5\kappa^5 e^{-r^2 \kappa^2}}{15\sqrt{\pi}} \right. \nonumber \\
 &+ \left. \frac{16r^7\kappa^7 e^{-r^2 \kappa^2}}{105\sqrt{\pi}} +  \mathrm{erfc(\kappa r)} \right)
\end{align}  

We can divide ${T}_{\alpha\beta\gamma\delta}(\mathbf{r})$ into three
parts:
\begin{eqnarray}
& & \mathbf{T}(\mathbf{r}) =
    \mathbf{T}^\mathrm{K}(\mathbf{r}) +
    \mathbf{T}^\mathrm{R1}(\mathbf{r}) +
    \mathbf{T}^\mathrm{R2}(\mathbf{r}) 
\end{eqnarray}
where the first term is the reciprocal space portion.  Since the
quadrupolar correction factor $B = \tilde{T}_{abab}(0) / 4\pi$ and
$\mathbf{k} = 0 $ is excluded from the reciprocal space sum,
$\mathbf{T}^\mathrm{K}$ will not contribute.\cite{NeumannII83} The
remaining terms,
\begin{align}
\mathbf{T}^\mathrm{R1}(\mathbf{r}) =&  \mathbf{T}^\mathrm{bare}(\mathbf{r}) \left(\frac{2r\kappa e^{-r^2
          \kappa^2}}{\sqrt{\pi}}+\frac{4r^3\kappa^3 e^{-r^2 \kappa^2}}{3\sqrt{\pi}}+\frac{8r^5\kappa^5 e^{-r^2 \kappa^2}}{15\sqrt{\pi}} \right. \nonumber \\
 &+ \left. \frac{16r^7\kappa^7 e^{-r^2 \kappa^2}}{105\sqrt{\pi}} +  \mathrm{erfc(\kappa r)} \right)
\end{align}
and
\begin{align}
T^\mathrm{R2}_{\alpha\beta\gamma\delta}(\mathbf{r}) =&  \left(\delta_{\gamma\delta} r_\alpha r_\beta +  \mathrm{ 5\; permutations}\right) \frac{16 \kappa^7 e^{-r^2 \kappa^2}}{7\sqrt{\pi}} \nonumber \\
& -\left(\delta_{\alpha\beta}\delta_{\gamma\delta}+\delta_{\alpha\gamma}\delta_{\beta\delta}+\delta_{\alpha\delta}\delta_{\beta\gamma}\right) \left(\frac{8 \kappa^5 e^{-r^2 \kappa^2}}{5\sqrt{\pi}}+ \frac{16 r^2\kappa^7 e^{-r^2 \kappa^2}}{35\sqrt{\pi} }\right)
\end{align}
are contributions from the real space
sum.\cite{Adams76,Adams80,Adams81} Here
$\mathbf{T}^\mathrm{bare}(\mathbf{r})$ is the unmodified quadrupolar
tensor (for undamped quadrupoles).  Due to the angular symmetry of the
unmodified tensor, the integral of
$\mathbf{T}^\mathrm{R1}(\mathbf{r})$ will vanish when integrated over
a spherical region. The only term contributing to the correction
factor (B) is therefore
$T^\mathrm{R2}_{\alpha\beta\gamma\delta}(\mathbf{r})$, which allows us
to derive the correction factor for the Ewald-Kornfeld (EK) method,
\begin{eqnarray}
\mathrm{B} &=& \frac{1}{4\pi} \int_V T^\mathrm{R2}_{abab}(\mathbf{r}) \nonumber \\
&=& -\frac{8r_c^3 \kappa^5 e^{-\kappa^2 r_c^2}}{15\sqrt{\pi}}
\end{eqnarray}  
which is essentially identical with the correction factor from the
direct spherical cutoff (SC) method.

\section{Generating Uniform Field Gradients}
One important task in carrying out the simulations mentioned in the
Chapter 4 was to generate uniform electric field gradients.  To do
this, we relied heavily on both the notation and results from Torres
del Castillo and Mend\'{e}z Garido.\cite{Torres-del-Castillo:2006uo}
In this work, tensors were expressed in Cartesian components, using at
times a dyadic notation. This proves quite useful for computer
simulations that make use of toroidal boundary conditions.

An alternative formalism uses the theory of angular momentum and
spherical harmonics and is common in standard physics texts such as
Jackson,\cite{Jackson98} Morse and Feshbach,\cite{Morse:1946zr} and
Stone.\cite{Stone:1997ly} Because this approach has its own
advantages, relationships are provided below comparing that
terminology to the Cartesian tensor notation.

The gradient of the electric field is given by
\begin{equation*}
\mathsf{G}(\mathbf{r}) = -\nabla \nabla \Phi(\mathbf{r}),
\end{equation*}
where $\Phi(\mathbf{r})$ is the electrostatic potential.  In a
charge-free region of space, $\nabla \cdot \mathbf{E}=0$, and
$\mathsf{G}$ is a symmetric traceless tensor.  From symmetry
arguments, we know that this tensor can be written in terms of just
five independent components.

Following Torres del Castillo and Mend\'{e}z Garido's notation, the
gradient of the electric field may also be written in terms of two
vectors $\mathbf{a}$ and $\mathbf{b}$,
\begin{equation*}
G_{ij}=\frac{1}{2} (a_i b_j + a_j b_i) - \frac{1}{3}(\mathbf a \cdot \mathbf b) \delta_{ij} .
\end{equation*} 
If the vectors $\mathbf{a}$ and $\mathbf{b}$ are unit vectors, the
electrostatic potential that generates a uniform gradient may be
written:
\begin{align}
\Phi(x, y, z) =\; -\frac{g_o}{2} & \left(\left(a_1b_1 -
                         \frac{cos\psi}{3}\right)\;x^2+\left(a_2b_2
                         - \frac{cos\psi}{3}\right)\;y^2 +
                         \left(a_3b_3 -
                         \frac{cos\psi}{3}\right)\;z^2 \right. \nonumber \\
 & + (a_1b_2 + a_2b_1)\; xy + (a_1b_3 + a_3b_1)\; xz + (a_2b_3 + a_3b_2)\; yz \bigg) .
\label{eq:appliedPotential}
\end{align} 
Note $\mathbf{a}\cdot\mathbf{a} = \mathbf{b} \cdot \mathbf{b} = 1$,
$\mathbf{a} \cdot \mathbf{b}=\cos \psi$, and $g_0$ is the overall
strength of the potential.

Taking the gradient of Eq.~(\ref{eq:appliedPotential}), we find the
field due to this potential,
\begin{equation}
\mathbf{E} = -\nabla \Phi
=\frac{g_o}{2} \left(\begin{array}{ccc}
2(a_1 b_1 - \frac{cos\psi}{3})\; x & +\; (a_1 b_2 + a_2 b_1)\; y & +\; (a_1 b_3 + a_3 b_1)\; z \\
 (a_2 b_1 + a_1 b_2)\; x & +\; 2(a_2 b_2 - \frac{cos\psi}{3})\; y & +\;  (a_2 b_3 + a_3 b_3)\; z \\
(a_3 b_1 + a_3 b_2)\; x & +\;  (a_3 b_2 + a_2 b_3)\; y & +\; 2(a_3 b_3 - \frac{cos\psi}{3})\; z 
\end{array} \right),
\label{eq:CE}
\end{equation} 
while the gradient of the electric field in this form,
\begin{equation}
\mathsf{G} = \nabla\mathbf{E} 
= \frac{g_o}{2}\left(\begin{array}{ccc}
2(a_1\; b_1 - \frac{cos\psi}{3}) &  (a_1\; b_2 \;+ a_2\; b_1) & (a_1\; b_3 \;+ a_3\; b_1) \\
 (a_2\; b_1 \;+ a_1\; b_2) & 2(a_2\; b_2 \;- \frac{cos\psi}{3}) & (a_2\; b_3 \;+ a_3\; b_3) \\
(a_3\; b_1 \;+ a_3\; b_2) & (a_3\; b_2 \;+ a_2\; b_3) & 2(a_3\; b_3 \;- \frac{cos\psi}{3})
\end{array} \right),
\label{eq:GC}
\end{equation}  
is uniform over the entire space.  Therefore, to describe a uniform
gradient in this notation, two unit vectors ($\mathbf{a}$ and
$\mathbf{b}$) as well as a potential strength, $g_0$, must be
specified. As expected, this requires five independent parameters.

The common alternative to the Cartesian notation expresses the
electrostatic potential using the notation of Morse and
Feshbach,\cite{Morse:1946zr}
\begin{equation} \label{eq:quad_phi} 
\Phi(x,y,z) = -\left[ a_{20} \frac{2 z^2 -x^2 - y^2}{2}
+ 3 a_{21}^e \,xz + 3 a_{21}^o \,yz  
 + 6a_{22}^e \,xy +  3 a_{22}^o (x^2 - y^2) \right].
\end{equation}
Here we use the standard $(l,m)$ form for the $a_{lm}$ coefficients,
with superscript $e$ and $o$ denoting even and odd, respectively.
This form makes the functional analogy to ``d'' atomic states
apparent. 

Applying the gradient operator to Eq. (\ref{eq:quad_phi}) the electric
field due to this potential,
\begin{equation}
\mathbf{E} = -\nabla \Phi
= \left(\begin{array}{ccc}
\left( 6a_{22}^o -a_{20} \right)\; x &+\; 6a_{22}^e\; y &+\; 3a_{21}^e\;  z  \\
6a_{22}^e\; x & -\; (a_{20} + 6a_{22}^o)\; y & +\; 3a_{21}^o\; z \\
3a_{21}^e\; x & +\; 3a_{21}^o\; y & +\; 2a_{20}\; z
\end{array} \right),
\label{eq:MFE}
\end{equation}
while the gradient of the electric field in this form is:
\begin{equation} \label{eq:grad_e2}
\mathsf{G} = 
\begin{pmatrix}
6 a_{22}^o - a_{20} & 6a_{22}^e & 3a_{21}^e\\
6a_{22}^e & -(a_{20}+6a_{22}^o) & 3a_{21}^o \\
3a_{21}^e  &  3a_{21}^o & 2a_{20} \\
\end{pmatrix} \\
\end{equation}
which is also uniform over the entire space.  This form for the
gradient can be factored as
\begin{gather}
\begin{aligned}
\mathsf{G}  = a_{20} 
\begin{pmatrix}
-1 & 0 & 0\\
0 & -1 & 0\\
0 & 0 & 2\\
\end{pmatrix}
+3a_{21}^e
\begin{pmatrix}
0 & 0 & 1\\
0 & 0 & 0\\
1 & 0 & 0\\
\end{pmatrix}
+3a_{21}^o
\begin{pmatrix}
0 & 0 & 0\\
0 & 0 & 1\\
0 & 1 & 0\\
\end{pmatrix} \\
+6a_{22}^e
\begin{pmatrix}
0 & 1 & 0\\
1 & 0 & 0\\
0 & 0 & 0\\
\end{pmatrix}
+6a_{22}^o
\begin{pmatrix}
1 & 0 & 0\\
0 & -1 & 0\\
0 & 0 & 0\\
\end{pmatrix}.
\end{aligned}
 \label{eq:intro_tensors}
\end{gather}
The five matrices in the expression above represent five different
symmetric traceless tensors of rank 2. 

It is useful to find the Cartesian vectors $\mathbf a$ and $\mathbf b$
that generate the five types of tensors shown in
Eq.~(\ref{eq:intro_tensors}).  If the two vectors are co-linear, e.g.,
$\psi=0$, $\mathbf{a}=(0,0,1)$ and $\mathbf{b}=(0,0,1)$, then
\begin{equation*}
\mathsf{G} = \frac{g_0}{3}
\begin{pmatrix}
-1 & 0 & 0 \\
0 & -1 & 0 \\
0 & 0 & 2 \\
\end{pmatrix} ,
\end{equation*}
which is the $a_{20}$ symmetry.
To generate the $a_{22}^o$ symmetry, we take:
$\mathbf{a}= (\frac{1}{\sqrt{2}}, \frac{1}{\sqrt{2}},0)$ and
$\mathbf{b}=(\frac{1}{\sqrt{2}}, -\frac{1}{\sqrt{2}},0)$
and find:
\begin{equation*}
\mathsf{G}=\frac{g_0}{2}
\begin{pmatrix}
1 & 0 & 0 \\
0 & -1 & 0 \\
0 & 0 & 0 \\
\end{pmatrix} .
\end{equation*}
To generate the $a_{22}^e$ symmetry, we take:
$\mathbf{a}= (1, 0, 0)$ and $\mathbf{b} = (0,1,0)$ and find:
\begin{equation*}
\mathsf{G}=\frac{g_0}{2}
\begin{pmatrix}
0 & 1 & 0 \\
1 & 0 & 0 \\
0 & 0 & 0 \\
\end{pmatrix} .
\end{equation*}
The pattern is straightforward to continue for the other symmetries.

We find the notation of Ref. \cite{Torres-del-Castillo:2006uo}
helpful when creating specific types of constant gradient electric
fields in simulations. For this reason,
Eqs. (\ref{eq:appliedPotential}), (\ref{eq:CE}), and (\ref{eq:GC}) are
implemented in our code.  In the simulations using constant applied
gradients that are mentioned in the main text, we utilized a field
with the $a_{22}^e$ symmetry using vectors, $\mathbf{a}= (1, 0, 0)$
and $\mathbf{b} = (0,1,0)$.

\section{Point-multipolar interactions with a spatially-varying electric field}

This section develops formulas for the force and torque exerted by an
external electric field, $\mathbf{E}(\mathbf{r})$, on object
$a$.\cite{Raab:2004ve} Object $a$ has an embedded collection of
charges and in simulations will represent a molecule, ion, or a
coarse-grained substructure. We describe the charge distributions
using primitive multipoles defined in Ref.~\cite{PaperI} by
\begin{align}
C_a =&\sum_{k \, \text{in }a} q_k , \label{eq:charge} \\
D_{a\alpha} =&\sum_{k \, \text{in }a} q_k r_{k\alpha}, \label{eq:dipole}\\
Q_{a\alpha\beta} =& \frac{1}{2} \sum_{k \, \text{in }  a} q_k
r_{k\alpha}  r_{k\beta},
\label{eq:quadrupole}
\end{align}
where $\mathbf{r}_k$ is the local coordinate system for the object
(usually the center of mass of object $a$).  Components of vectors and
tensors are given using the Einstein repeated summation notation. Note
that the definition of the primitive quadrupole here differs from the
standard traceless form, and contains an additional Taylor-series
based factor of $1/2$. In Ref.~\cite{PaperI}, we derived the
forces and torques each object exerts on the other objects in the
system.

Here we must also consider an external electric field that varies in
space: $\mathbf E(\mathbf r)$.  Each of the local charges $q_k$ in
object $a$ will then experience a slightly different field.  This
electric field can be expanded in a Taylor series around the local
origin of each object. For a particular charge $q_k$, the electric
field at that site's position is given by:
\begin{equation}
\mathbf{E}(\mathbf{r}_k) = E_\gamma|_{\mathbf{r}_k = 0} + \nabla_\delta E_\gamma |_{\mathbf{r}_k = 0}  r_{k \delta} 
+ \frac {1}{2} \nabla_\delta \nabla_\varepsilon E_\gamma|_{\mathbf{r}_k = 0}  r_{k \delta}
r_{k \varepsilon} + ... 
\end{equation}
Note that if one shrinks object $a$ to a single point, the
${E}_\gamma$ terms are all evaluated at the center of the object (now
a point). Thus later the ${E}_\gamma$ terms can be written using the
same (molecular) origin for all point charges in the object. The force
exerted on object $a$ by the electric field is given by,
\begin{align}
F^a_\gamma = \sum_{k \textrm{~in~} a} q_k E_\gamma(\mathbf{r}_k) &=  \sum_{k \textrm{~in~} a} q_k \lbrace E_\gamma + \nabla_\delta E_\gamma r_{k \delta} 
+ \frac {1}{2} \nabla_\delta \nabla_\varepsilon E_\gamma r_{k \delta}
r_{k \varepsilon} + ...  \rbrace  \\
 &= C_a E_\gamma + D_{a  \delta} \nabla_\delta E_\gamma 
+ Q_{a \delta \varepsilon} \nabla_\delta \nabla_\varepsilon E_\gamma +
... 
\end{align}
Thus in terms of the global origin $\mathbf{r}$, ${F}_\gamma(\mathbf{r}) = C {E}_\gamma(\mathbf{r})$ etc. 
  
Similarly, the torque exerted by the field on $a$ can be expressed as
\begin{align}
\tau^a_\alpha &=  \sum_{k \textrm{~in~} a} (\mathbf r_k \times q_k \mathbf E)_\alpha \\
 & =  \sum_{k \textrm{~in~} a} \epsilon_{\alpha \beta \gamma} q_k
 r_{k\beta} E_\gamma(\mathbf r_k) \\
 & = \epsilon_{\alpha \beta \gamma} D_\beta E_\gamma 
+ 2 \epsilon_{\alpha \beta \gamma} Q_{\beta \delta} \nabla_\delta
E_\gamma + ...
\end{align}
We note that the Levi-Civita symbol can be eliminated by utilizing the matrix cross product as defined in Ref.~\cite{Smith98}:
\begin{equation}
\left[\mathsf{A} \times \mathsf{B}\right]_\alpha = \sum_\beta
\left[\mathsf{A}_{\alpha+1,\beta} \mathsf{B}_{\alpha+2,\beta}
  -\mathsf{A}_{\alpha+2,\beta} \mathsf{B}_{\alpha+1,\beta} 
\right]
\label{eq:matrixCross}
\end{equation}
where $\alpha+1$ and $\alpha+2$ are regarded as cyclic permuations of
the matrix indices. Finally, the interaction energy $U^a$ of object $a$ with the external field is given by,
\begin{equation}
U^a = \sum_{k~in~a} q_k \phi_k (\mathrm{r}_k)
\end{equation}
Performing another Taylor series expansion about the local body origin,
\begin{equation}
\phi({\mathbf{r}_k}) = \phi|_{\mathbf{r}_k = 0 } + r_{k \alpha} \nabla_\alpha \phi_\alpha|_{\mathbf{r}_k = 0 } + \frac{1}{2} r_{k\alpha}r_{k\beta}\nabla_\alpha \nabla_\beta \phi|_{\mathbf{r}_k = 0} + ...
\end{equation}
Writing this in terms of the global origin $\mathbf{r}$, we find
\begin{equation}
U(\mathbf{r}) = \mathrm{C} \phi(\mathbf{r}) - \mathrm{D}_\alpha \mathrm{E}_\alpha - \mathrm{Q}_{\alpha\beta}\nabla_\alpha \mathrm{E}_\beta + ...
\end{equation}
These results have been summarized in Table \ref{tab:UFT}.

\begin{table}
\begin{center}
\caption{Potential energy $(U)$, force $(\mathbf{F})$, and torque
  $(\mathbf{\tau})$ expressions for a multipolar site at $\mathbf{r}$ in an
  electric field, $\mathbf{E}(\mathbf{r})$ using the definitions of the multipoles in Eqs. (\ref{eq:charge}), (\ref{eq:dipole}) and (\ref{eq:quadrupole}).  
  \label{tab:UFT}}
\begin{tabular}{r|C{3cm}C{3cm}C{3cm}}
  & Charge & Dipole & Quadrupole \\ \hline
$U(\mathbf{r})$ &  $C \phi(\mathbf{r})$ & $-\mathbf{D} \cdot \mathbf{E}(\mathbf{r})$ & $- \mathsf{Q}:\nabla \mathbf{E}(\mathbf{r})$ \\
$\mathbf{F}(\mathbf{r})$ & $C \mathbf{E}(\mathbf{r})$ & $\mathbf{D} \cdot \nabla \mathbf{E}(\mathbf{r})$ &  $\mathsf{Q} : \nabla\nabla\mathbf{E}(\mathbf{r})$ \\
$\mathbf{\tau}(\mathbf{r})$ & & $\mathbf{D} \times \mathbf{E}(\mathbf{r})$ & $2 \mathsf{Q} \times \nabla \mathbf{E}(\mathbf{r})$
\end{tabular}
\end{center}
\end{table}

\section{Boltzmann averages for orientational polarization}
If we consider a collection of molecules in the presence of external
field, the perturbation experienced by any one molecule will include
contributions to the field or field gradient produced by the all other
molecules in the system. In subsections
\ref{subsec:boltzAverage-Dipole} and \ref{subsec:boltzAverage-Quad},
we discuss the molecular polarization due solely to external field
perturbations.  This illustrates the origins of the polarizability
Eqs.~(~\ref{flucDipole},\ref{pertDipole}, \ref{pertQuad} and \ref{flucQuad}).

\subsection{Dipoles}
\label{subsec:boltzAverage-Dipole}
Consider a system of molecules, each with permanent dipole moment
$p_o$. In the absence of an external field, thermal agitation orients
the dipoles randomly, and the system moment, $\mathbf{P}$, is zero.
External fields will line up the dipoles in the direction of applied
field.  Here we consider the net field from all other molecules to be
zero.  Therefore the total Hamiltonian acting on each molecule
is,\cite{Jackson98}
\begin{equation}
H = H_o - \mathbf{p}_o \cdot \mathbf{E},
\end{equation}
where $H_o$ is a function of the internal coordinates of the molecule.
The Boltzmann average of the dipole moment in the direction of the
field is given by,
\begin{equation}
\langle p_{mol} \rangle = \frac{\displaystyle\int p_o \cos\theta
  e^{~p_o E \cos\theta /k_B T}\; d\Omega}{\displaystyle\int  e^{~p_o E \cos\theta/k_B
    T}\; d\Omega},
\end{equation}
where the $z$-axis is taken in the direction of the applied field,
$\bf{E}$ and
$\int d\Omega = \int_0^\pi \sin\theta\; d\theta \int_0^{2\pi} d\phi
\int_0^{2\pi} d\psi$
is an integration over Euler angles describing the orientation of the
molecule.

If the external fields are small, \textit{i.e.}
$p_oE \cos\theta / k_B T << 1$,
\begin{equation}
\langle p_{mol} \rangle \approx \frac{{p_o}^2}{3 k_B T}E,
\end{equation}
where $ \alpha_p = \frac{{p_o}^2}{3 k_B T}$ is the molecular
polarizability. The orientational polarization depends inversely on
the temperature as the applied field must overcome thermal agitation
to orient the dipoles.

\subsection{Quadrupoles}
\label{subsec:boltzAverage-Quad}
If instead, our system consists of molecules with permanent
\textit{quadrupole} tensor $q_{\alpha\beta}$. The average quadrupole
at temperature $T$ in the presence of uniform applied field gradient
is given by,\cite{AduGyamfi78, AduGyamfi81}
\begin{equation}
\langle q_{\alpha\beta} \rangle \;=\; \frac{\displaystyle\int
  q_{\alpha\beta}\; e^{-H/k_B T}\; d\Omega}{\displaystyle\int
  e^{-H/k_B T}\; d\Omega} \;=\; \frac{\displaystyle\int
  q_{\alpha\beta}\; e^{~q_{\mu\nu}\;\partial_\nu E_\mu /k_B T}\;
  d\Omega}{\displaystyle\int  e^{~q_{\mu\nu}\;\partial_\nu E_\mu /k_B
    T}\; d\Omega },
\label{boltzQuad}
\end{equation}
where $H = H_o - q_{\mu\nu}\;\partial_\nu E_\mu $ is the energy of a
quadrupole in the gradient of the applied field and $H_o$ is a
function of internal coordinates of the molecule. The energy and
quadrupole moment can be transformed into the body frame using a
rotation matrix $\mathsf{\eta}^{-1}$,
\begin{align}
q_{\alpha\beta} &= \eta_{\alpha\alpha'}\;\eta_{\beta\beta'}\;{q}^* _{\alpha'\beta'} \\
H &= H_o - q:{\nabla}\mathbf{E} \\
  &= H_o - q_{\mu\nu}\;\partial_\nu E_\mu  \\
  &= H_o
    -\eta_{\mu\mu'}\;\eta_{\nu\nu'}\;{q}^*_{\mu'\nu'}\;\partial_\nu
    E_\mu. \label{energyQuad}
\end{align}
Here the starred tensors are the components in the body fixed
frame. Substituting Eq.~(\ref{energyQuad}) in the Eq.~(\ref{boltzQuad}) and taking linear terms in the expansion we obtain,
\begin{equation}
\braket{q_{\alpha\beta}} = \frac{\displaystyle \int q_{\alpha\beta} \left(1 +
    \frac{\eta_{\mu\mu'}\;\eta_{\nu\nu'}\;{q}^*_{\mu'\nu'}\;\partial_\nu
      E_\mu }{k_B T}\right)\;  d\Omega}{\displaystyle \int \left(1 + \frac{\eta_{\mu\mu'}\;\eta_{\nu\nu'}\;{q}^*_{\mu'\nu'}\;\partial_\nu E_\mu }{k_B T}\right)\; d\Omega}.
\end{equation}
Note that $\eta_{\alpha\alpha'}$ is the inverse of the rotation
matrix that transforms the body fixed coordinates to the space
coordinates.
% \[\eta_{\alpha\alpha'} 
% = \left(\begin{array}{ccc}
% cos\phi\; cos\psi - cos\theta\; sin\phi\; sin\psi & -cos\theta\; cos\psi\; sin\phi - cos\phi\; sin\psi & sin\theta\; sin\phi \\
% cos\psi\; sin\phi + cos\theta\; cos\phi \; sin\psi & cos\theta\; cos\phi\; cos\psi - sin\phi\; sin\psi & -cos\phi\; sin\theta \\
% sin\theta\; sin\psi & -cos\psi\; sin\theta & cos\theta
% \end{array} \right).\]

Integration of the first and second terms in the denominator gives
$8 \pi^2$ and
$8 \pi^2 ({\nabla} \cdot \mathbf{E}) \mathrm{Tr}(q^*) / 3 $
respectively. The second term vanishes for charge free space (where
${\nabla} \cdot \mathbf{E}=0$). Similarly, integration of the first
term in the numerator produces
$8 \pi^2 \delta_{\alpha\beta} \mathrm{Tr}(q^*) / 3$ and second
produces
$8 \pi^2 (3{q}^*_{\alpha'\beta'}{q}^*_{\beta'\alpha'} -
{q}^*_{\alpha'\alpha'}{q}^*_{\beta'\beta'})\partial_\alpha E_\beta /
15 k_B T $.
Therefore the Boltzmann average of a quadrupole moment can be written
as,
\begin{equation}
\langle q_{\alpha\beta} \rangle =  \frac{1}{3} \mathrm{Tr}(q^*)\;\delta_{\alpha\beta} + \frac{{\bar{q_o}}^2}{15k_BT}\;\partial_\alpha E_\beta,
\end{equation}
where $\alpha_q = \frac{{\bar{q_o}}^2}{15k_BT} $ is a molecular
quadrupole polarizablity and
${\bar{q_o}}^2=
3{q}^*_{\alpha'\beta'}{q}^*_{\beta'\alpha'}-{q}^*_{\alpha'\alpha'}{q}^*_{\beta'\beta'}$
is the square of the net quadrupole moment of a molecule.

\section{Gradient of the field due to quadrupolar polarization}
\label{singularQuad}
In section IV.C of the Chapter 4, we stated that for quadrupolar
fluids, the self-contribution to the field gradient vanishes at the
singularity. In this section, we prove this statement.  For this
purpose, we consider a distribution of charge $\rho(\mathbf{r})$ which
gives rise to an electric field $\mathbf{E}(\mathbf{r})$ and gradient
of the field $\nabla\mathbf{E}(\mathbf{r})$ throughout space. The
gradient of the electric field over volume due to the charges within
the sphere of radius $R$ is given by (cf. Ref.~\cite{Jackson98},
Eq.~4.14):
\begin{equation}
\int_{r<R} \nabla\mathbf{E} d\mathbf{r} = -\int_{r=R} R^2 \mathbf{E}\;\hat{n}\; d\Omega
\label{eq:8}
\end{equation}
where $d\Omega$ is the solid angle and $\hat{n}$ is the normal vector
of the surface of the sphere, 
\begin{equation}
\hat{n} = \sin\theta\cos\phi\; \hat{x} + \sin\theta\sin\phi\; \hat{y} +
\cos\theta\; \hat{z}
\end{equation}
in spherical coordinates.  For the charge density $\rho(\mathbf{r}')$, the
total gradient of the electric field can be written as,\cite{Jackson98}
\begin{equation}
\int_{r<R} {\nabla}\mathbf {E}\; d\mathbf{r}=-\int_{r=R} R^2\;
{\nabla}\Phi\; \hat{n}\; d\Omega
=-\frac{1}{4\pi\;\epsilon_o}\int_{r=R} R^2\; {\nabla}\;\left(\int
  \frac{\rho(\mathbf
    r')}{|\mathbf{r}-\mathbf{r}'|}\;d\mathbf{r}'\right) \hat{n}\;
d\Omega .
\label{eq:9}
\end{equation}
The radial function in the Eq.~(\ref{eq:9}) can be expressed in
terms of spherical harmonics as,\cite{Jackson98}
\begin{equation}
\frac{1}{|\mathbf{r} - \mathbf{r}'|} = 4\pi \sum_{l=0}^{\infty}\sum_{m=-l}^{m=l}\frac{1}{2l+1}\;\frac{{r^l_<}}{{r^{l+1}_>}}\;{Y^*}_{lm}(\theta', \phi')\;Y_{lm}(\theta, \phi)
\label{eq:10}
\end{equation}
If the sphere completely encloses the charge density then $ r_< = r'$ and $r_> = R$. Substituting Eq.~(\ref{eq:10}) into (\ref{eq:9}) we get,
\begin{align}
\int_{r<R} {\nabla}\mathbf{E}\;d\mathbf{r} &=-\frac{R^2}{\epsilon_o}\int_{r=R} \; {\nabla}\;\Bigl(\int \rho(\mathbf r')\sum_{l=0}^{\infty}\sum_{m=-l}^{m=l}\frac{1}{2l+1}\;\frac{{r'^l}}{{R^{l+1}}}\;  \nonumber \\
&  {Y^*}_{lm}(\theta', \phi')\;Y_{lm}(\theta, \phi)\;d\mathbf{r}'\Bigr) \hat{n}\; d\Omega \nonumber \\
 &= -\frac{R^2}{\epsilon_o}\sum_{l=0}^{\infty}\sum_{m=-l}^{m=l}\frac{1}{2l+1}\;\int \rho(\mathbf r')\;{r'^l}\;{Y^*}_{lm}(\theta', \phi') \nonumber \\
 &\Bigl(\int_{r=R}\vec{\nabla}\left({R^{-(l+1)}}\;Y_{lm}(\theta, \phi)\right)\hat{n}\; d\Omega \Bigr)d\mathbf{r}
' .
\label{eq:11}
\end{align} 
The gradient of the product of radial function and spherical harmonics
is given by:\cite{Arfkan}
\begin{equation}
\begin{split}
{\nabla}\left[ f(r)\;Y_{lm}(\theta, \phi)\right] =& -\left(\frac{l+1}{2l+1}\right)^{1/2}\; \left[\frac{\partial}{\partial r}-\frac{l}{r} \right]f(r)\; Y_{l, l+1, m}(\theta, \phi)\\ &+ \left(\frac{l}{2l+1}\right)^{1/2}\left[\frac
{\partial}{\partial r}+\frac{l}{r} \right]f(r)\; Y_{l, l-1, m}(\theta, \phi).
\end{split}
\label{eq:12}
\end{equation}
where $Y_{l,l+1,m}(\theta, \phi)$ is a vector spherical
harmonic.\cite{Arfkan} Using Eq.~(\ref{eq:12}) we get,
\begin{equation}
{\nabla}\left({R^{-(l+1)}}\;Y_{lm}(\theta, \phi)\right) = [(l+1)(2l+1)]^{1/2}\; Y_{l,l+1,m}(\theta, \phi) \; \frac{1}{R^{l+2}},
\label{eq:13}
\end{equation}
Using Clebsch-Gordan coefficients $C(l+1,1,l|m_1,m_2,m)$, the vector
spherical harmonics can be written in terms of spherical harmonics,
\begin{equation}
Y_{l,l+1,m}(\theta, \phi) = \sum_{m_1, m_2} C(l+1,1,l|m_1,m_2,m)\; Y_{l+1}^{m_1}(\theta,\phi)\; \hat{e}_{m_2}.
\label{eq:14}
\end{equation}
Here $\hat{e}_{m_2}$ is a spherical tensor of rank 1 which can be expressed
in terms of Cartesian coordinates,
\begin{equation}
{\hat{e}}_{+1} = - \frac{\hat{x}+i\hat{y}}{\sqrt{2}},\quad {\hat{e}}_{0} = \hat{z},\quad and \quad {\hat{e}}_{-1} = \frac{\hat{x}-i\hat{y}}{\sqrt{2}}.
\label{eq:15}
\end{equation} 
The normal vector $\hat{n} $ is then expressed in terms of spherical tensor of rank 1 as shown in below,
\begin{equation}
\hat{n} = \sqrt{\frac{4\pi}{3}}\left(-Y_1^{-1}{\hat{e}}_1 - Y_1^{1}{\hat{e}}_{-1} + Y_1^{0}{\hat{e}}_0 \right).
\label{eq:16}
\end{equation}
The surface integral of the product of $\hat{n}$ and
$Y_{l+1}^{m_1}(\theta, \phi)$ gives,
\begin{equation}
\begin{split}
\int \hat{n}\;Y_{l+1}^{m_1}\;d\Omega &= \int \sqrt{\frac{4\pi}{3}}\left(-Y_1^{-1}{\hat{e}}_1 -Y_1^{1}{\hat{e}}_{-1} + Y_1^{0}{\hat{e}}_0 \right)\;Y_{l+1}^{m_1}\; d\Omega \\
&=  \int \sqrt{\frac{4\pi}{3}}\left({Y_1^{1}}^* {\hat{e}}_1 +{Y_1^{-1}}^* {\hat{e}}_{-1} + {Y_1^{0}}^* {\hat{e}}_0 \right)\;Y_{l+1}^{m_1}\; d\Omega \\
&=   \sqrt{\frac{4\pi}{3}}\left({\delta}_{l+1, 1}\;{\delta}_{1, m_1}\;{\hat{e}}_1 + {\delta}_{l+1, 1}\;{\delta}_{-1, m_1}\;{\hat{e}}_{-1}+ {\delta}_{l+1, 1}\;{\delta}_{0, m_1} \;{\hat{e}}_0\right),
\end{split}
\label{eq:17}
\end{equation}
where $Y_{l}^{-m} = (-1)^m\;{Y_{l}^{m}}^* $ and
$ \int {Y_{l}^{m}}^* Y_{l'}^{m'}\;d\Omega =
\delta_{ll'}\delta_{mm'} $.
Non-vanishing values of Eq.~\ref{eq:17} require $l = 0$,
therefore the value of $ m = 0 $. Since the values of $ m_1$ are -1,
1, and 0 then $m_2$ takes the values 1, -1, and 0, respectively
provided that $m = m_1 + m_2$.  Eq.~\ref{eq:11} can therefore be
modified,
\begin{equation}
\begin{split}
\int_{r<R} {\nabla}\mathbf{E}\;d\mathbf{r} = &- \sqrt{\frac{4\pi}{{3}}}\;\frac{1}{\epsilon_o}\int \rho(r')\;{Y^*}_{00}(\theta', \phi')[ C(1, 1, 0|-1,1,0)\;{\hat{e}_{-1}}{\hat{e}_{1}}\\  &+ C(1, 1, 0|-1,1,0)\;{\hat{e}_{1}}{\hat{e}_{-1}}+C(
1, 1, 0|0,0,0)\;{\hat{e}_{0}}{\hat{e}_{0}} ]\; d\mathbf{r}' \\
&= -\sqrt{\frac{4\pi}{{3}}}\;\frac{1}{\epsilon_o}\int \rho(r')\;d\mathbf{r}'\left({\hat{e}_{-1}}{\hat{e}_{1}}+{\hat{e}_{1}}{\hat{e}_{-1}}-{\hat{e}_{0}}{\hat{e}_{0}}\right)\\
&= - \sqrt{\frac{4\pi}{{3}}}\;\frac{1}{\epsilon_o}\;C_\mathrm{total}\;\left({\hat{e}_{-1}}{\hat{e}_{1}}+{\hat{e}_{1}}{\hat{e}_{-1}}-{\hat{e}_{0}}{\hat{e}_{0}}\right).
\end{split}
\label{eq:19} 
\end{equation}
In the last step, the charge density was integrated over the sphere,
yielding a total charge $C_\mathrm{total}$. Eq.~(\ref{eq:19})
gives the total gradient of the field over a sphere due to the
distribution of the charges.  For quadrupolar fluids the total charge
within a sphere is zero, therefore
$ \int_{r<R} {\nabla}\mathbf{E}\;d\mathbf{r} = 0 $.  Hence the quadrupolar
polarization produces zero net gradient of the field inside the
sphere.

\section{Quadrupole tensors and electric field gradients}

In this section of this appendix, useful relationships for quadrupole tensors and electric field gradients are derived.
We use heavily both the notation and results from Torres del Castillo and Mend\'{e}z Garido (Ref.~\cite{Torres-del-Castillo:2006uo}).  
In this reference, tensors are expressed in Cartesian components, using at times a dyadic notation.  
This proves to be useful for our work since in simulations, we frequently will want to use
toroidal boundary conditions, which are much more easily implemented using Cartesian coordinate
systems.  

An alternative formalism uses the theory of angular momentum and spherical harmonics;  e.g., see standard physics texts such as Jackson, Morse and Feshbach. Because this approach has its own advantages, relationships are provided below comparing that terminology to the tensor notation.

\subsection{General results}

In calculations, we have defined both a traced quadrupole tensor
\begin{equation*}
Q_{\alpha \beta} = \frac{1}{2} \sum_k q_k r_{k \alpha} r_{k \beta} = 
\frac{1}{2} \int r^{\prime}_i r^{\prime}_j  \rho(\mathbf{r}^{\prime}) d^3 r^{\prime} 
\end{equation*}
and a traceless form:
\begin{equation*}
\Theta_{\alpha \beta}=3Q_{\alpha \beta} - \delta_{\alpha \beta}\textrm{Tr} \mathsf{Q} .
\end{equation*} 
Here, we continue to use notation and results from Torres del Castillo and Mendez Gardio \cite{Torres-del-Castillo:2006uo}
\begin{equation*}
Q^C_{ij}=\int(3r^{\prime}_i r^{\prime}_j - r^{{\prime}2}\delta_{ij}) \rho(\mathbf{r}^{\prime}) d^3 r^{\prime}=2\Theta_{ij}  ,
\end{equation*}
in order to discuss generalized forms of traceless tensors $\mathsf{Q}^C$ 
and the gradients of the electric field.
In their notation, 
\begin{equation*}
\mathsf{G}=-\nabla _i \nabla_j \Phi(\mathbf{r})|_{\mathbf{r}=0}  ,
\end{equation*}
where $\Phi(\mathbf{r})$ is the electrostatic potential.  
Since in a charge-free region of space, $\nabla \cdot \mathbf{E}=0$, both
$\mathsf{Q}^C$  and $\mathsf{G}$  are symmetric traceless tensors.  From symmetry
arguments, we know that each tensor can be written in terms of just five independent components.
We choose these when needed to be  $Q^C_{11}$, $Q^C_{22}$, $Q^C_{12}$, $Q^C_{13}$, and $Q^C_{23}$ and
$G_{11}$, $G_{22}$, $G_{12}$, $G_{13}$, and $G_{23}$.
A better way to characterize these tensors is in terms of their eigenvalues, which are all real, 
and eigenvectors, which  are real and can be made to be orthonormal.  Following Ref.~\cite{Torres-del-Castillo:2006uo}, the eigenvalues and eigenvectors of 
$\mathsf{Q}$  are written respectively in the order $\lambda$, $\mu$, and $\nu$ and $\hat{X}$, $\hat{Y}$, and $\hat{Z}$,
where the eigenvalues are named in the order $\lambda \ge \nu \ge \mu$.  Thus a preferred plane $XY$ is singled out by the largest
and smallest eigenvalue.  A similar notation (with primes) is used for $\mathsf{G}$, $\lambda^\prime \ge \nu^\prime \ge \mu^\prime$.
Thus the traceless condition is written $\lambda+\mu+\nu=0$ and $\lambda^\prime+\mu^\prime+\nu^\prime=0$  and the 
two tensors are represented by:
\begin{equation*}
Q_{ij}^C=\lambda \hat{X}_i  \hat{X}_j + \mu \hat{Y}_i \hat{Y}_j + \nu \hat{Z}_i  \hat{Z}_j
\qquad \textrm{and} \qquad
G_{ij}=\lambda^\prime \hat{X}^\prime_i  \hat{X}^\prime_j 
+ \mu^\prime \hat{Y}^\prime_i \hat{Y}^\prime_j + \nu^\prime \hat{Z}^\prime_i  \hat{Z}_j^\prime  .
\end{equation*}

Using again the notation of Ref.~\cite{Torres-del-Castillo:2006uo}, the interaction energy of a quadrupole $\mathsf{Q}^C$ with the gradient of the 
electric field $\mathsf{G}$ is
\begin{equation*}
U=-\frac{1}{6}\sum_{i,j} G_{ij} Q^C_{ij}  .
\end{equation*}
Assuming that $\mathsf{G}$ in written in the space frame, 
generally one would need to transform $\mathsf{Q}^C$
from body to space frame:
\begin{equation*}
Q^C_{ij} = \eta_{i i^{\prime}} \eta_{j j^{\prime}} Q_{i^{\prime} j^{\prime}}^{C*}  .
\end{equation*}
Natural coordinates for the body frame of $\mathsf{Q}^C$ are its eigenvectors.

Now a single quadrupole can be considered to be embedded in the gradient of an electric field.
Torres del Castillo and Mend\'{e}z  Garido have shown that the lowest energy state is the one where the eigenvector axes align with each other;
$\hat{X}$ with $\hat{X}^\prime$, $\hat{Y}$ with $\hat{Y}^\prime$, and $\hat{Z}$ with $\hat{Z}^\prime$.
For that case of idealized alignment, tensor $\mathsf{Q}^C$ will be in its body frame.  Thus
\begin{equation*}
U=-\frac{1}{6} (\lambda \lambda^\prime + \mu \mu^\prime +\nu \nu^\prime)  .
\end{equation*}
This equation is used below in the derivation of the quadrupolar scaling factor $B$.

\subsection{Special cases of quadrupole tensors}

Any general quadrupole can be written in terms of 
position vectors $\mathbf{c}$ and $\mathbf{d}$ which 
go from one negative charge $-q$ to the two positive charges $q$, thus forming two of the sides of a parallelogram:
\begin{equation*}
Q^C_{ij}=3q(c_i d_j + c_j d_i) -2q(\mathbf c \cdot \mathbf d) \delta_{ij} .
\end{equation*}
Placing the origin a the center of the parallelogram that is in the $xy$ plane, 
charges $-q$ and $q$ are located at
$\pm(\mathbf{c}+\mathbf{d})/2$ and $\pm(\mathbf{c}-\mathbf{d})/2$, respectively;
\begin{equation*}
\mathsf{Q}^C = 2q
\begin{pmatrix}
2c_xd_x - c_yd_y & 3(c_xd_y + c_yd_x)& 0 \\
3(c_xd_y + c_yd_x) & 2c_yd_y - c_xd_x& 0 \\
0 & 0 & -(c_xd_x+c_yd_y)\\
\end{pmatrix}  .
\end{equation*}
As discussed previously, this tensor will have three distinct eigenvalues;  
in Ref.~\cite{Torres-del-Castillo:2006uo} this is rewritten it by eliminating the middle eigenvalue $\nu$ and its eigenvector $\hat{Z}$,
which leads to the form
\begin{equation*}
Q^C_{ij}=(2\lambda+\mu) \hat{X}_i \hat{X}_j + (2\mu + \lambda) \hat{Y}_i \hat{Y}_j
-(\lambda + \mu)\delta_{ij} .
\end{equation*}
Continuing to use Ref.~\cite{Torres-del-Castillo:2006uo} notation, vectors $\mathbf{v}$ and $\mathbf{w}$ are defined:
\begin{equation*}
\mathbf{v}=\sqrt{2\lambda+\mu}\,\hat{X} +\sqrt{-2\mu-\lambda} \, \hat{Y}
\qquad \textrm{and} \qquad 
\mathbf{w}=\sqrt{2\lambda+\mu}\,\hat{X} -\sqrt{-2\mu-\lambda}\,\hat{Y}  .
\end{equation*}
Now $|\mathbf{v}|=|\mathbf{w}|=\lambda-\mu$ and the angle between $\mathbf{v}$
and $\mathbf{w}$ is $\frac{3(\lambda+\nu)}{(\lambda-\nu)}$. 
Rewriting vectors $\mathbf{c}$ and $\mathbf{d}$ in rescaled form as 
\begin{equation*}
\mathbf{v}= \Big ( 6q\frac{|\mathbf{d}|}{|\mathbf{c}|} \Big )^{1/2} \mathbf{c}
\qquad \textrm{and} \qquad  
\mathbf{w}= \Big ( 6q\frac{|\mathbf{c}|}{|\mathbf{d}|} \Big )^{1/2} \mathbf{d} ,
\end{equation*}
one finds:
\begin{equation*}
Q^C_{ij}=\frac{1}{2} (c_i d_j + c_j d_i) - \frac{1}{3}(\mathbf c \cdot \mathbf d) \delta_{ij} .
\end{equation*}

Discussing special cases now, if vectors $\mathbf{v}$ and $\mathbf{w}$
are orthogonal, then $\lambda=-\nu$.  Using axes as defined above, 
this represents a square planar quadrupole which is located in the $XY$ plane where
\begin{equation*}
\mathsf{Q}^C=
\begin{pmatrix}
\lambda &0 & 0 \\
0 & -\lambda & 0 \\
0 & 0 & 0\\
\end{pmatrix} .
\end{equation*}
In a similar way,
a linear quadrupole with a natural body axis in the $\hat{Z}$ direction can be written:
\begin{equation*}
\mathsf{Q}^C=
\begin{pmatrix}
\nu &0 & 0 \\
0 & \nu & 0 \\
0 & 0 & -2\nu\\
\end{pmatrix} 
\qquad \textrm{or} \qquad
\mathsf{Q}^C=
\begin{pmatrix}
2\nu &0 & 0 \\
0 & -\nu & 0 \\
0 & 0 & -\nu\\
\end{pmatrix} .
\end{equation*}

If needed, the tensor components  $Q^C_{11}$, $Q^C_{22}$, $Q^C_{12}$, $Q^C_{13}$, and $Q^C_{23}$
can be written in terms of the five multipole moments associated with $l=2$ (Ref.~\cite{Jackson98}):
\begin{equation*}
q_{lm}=\int Y^*_{lm}(\theta^{\prime}, \phi^{\prime})r^{{\prime}l} \rho(\mathbf{r}^{\prime}) d^3 r^{\prime} .
\end{equation*}

\section{Alternative method for calculating correction factor $\mathrm{B}$}

Here, the method of calculation the quadrupole correction factor $B$ is explained.  We start from Eq.\ref{fourierQuad}, which reads:
\begin{equation}
\frac{1}{3} \Theta_{\alpha \beta}(\mathbf{r}) = \epsilon_0 \chi_Q
\Big (\partial_\alpha E_\beta^0(\mathbf{r}) + \frac{1}{24 \pi \epsilon_0}
\int_{|\mathbf{r}-\mathbf{r}^\prime| > 0}
T_{\alpha \beta \gamma \delta} (|\mathbf{r}-\mathbf{r}^\prime|) 
\Theta_{\gamma \delta} (\mathbf{r}) d\mathbf{r}^\prime \Big ).
\label{eq:p34}
\end{equation}
We reason in analogy to the dipole case.  For dipoles, one
considers a constant electric field oriented in the $\hat{z}$ direction.  
Then the dipole correction factor A, Eq.~(\ref{dipCorrFactor}), is derived.
For quadrupoles, we must consider a constant gradient electric field and
its effect of the orientation of an isolated quadrupole.  This can be done
using the dyadic forms for both $\Theta_{\alpha,\beta}$ and $\partial_\alpha E_\beta^0$, see Sec.~\ref{sec:corrFactor}. Using that  notation, 
$\Theta=\lambda \hat{X}  \hat{X} + \mu \hat{Y} \hat{Y} + \nu \hat{Z}  \hat{Z}$
and $\nabla \mathbf{E}^0=
\lambda^\prime \hat{X}^\prime  \hat{X}^\prime
+ \mu^\prime \hat{Y}^\prime \hat{Y}^\prime + \nu^\prime \hat{Z}^\prime  \hat{Z}^\prime$. The quadrupole orients itself into its lowest energy state; Cartesian axes 
$\hat{X}$, $\hat{Y}$, and $\hat{Z}$ and $\hat{X}^\prime$, $\hat{Y}^\prime$, and $\hat{Z}^\prime$ align,
as explained previously. Then both $\Theta_{\alpha,\beta}$ and $\partial_\alpha E_\beta^0$ are diagonal
and can be written simply in terms of their eigenvalues.

Now the contraction $T_{\alpha \beta \gamma \delta} \Theta_{\gamma \delta}$ can be more easily considered.  
The only components of $T_{\alpha \beta \gamma \delta}$ that are required are the nine terms with
$(\alpha,\beta) = (11), (22), (33)$ and $(\gamma,\delta)=(11), (22), (33)$.
Because of the symmetry of $T_{\alpha \beta \gamma \delta}$, the nine terms can be used to write a
symmetric matrix:
\begin{equation*}
\mathsf{M}(R)=
\begin{pmatrix}
{T}_{1111}(R) & {T}_{1122}(R) & {T}_{1133}(R)\\
{T}_{2211}(R) & {T}_{2222}(R) & {T}_{2233}(R)\\
{T}_{3311}(R) & {T}_{3322}(R) & {T}_{3333}(R)\\
\end{pmatrix}
\end{equation*}
where $\mathbf{R}=\mathbf{r}- \mathbf{r^\prime}$, $R=|\mathbf{r}- \mathbf{r^\prime}|$.\
and $\mathbf{R}_1=X$, etc.
Taking for example the case of a point quadrupole (Eq.~\ref{quadRadial} of the paper):
\begin{gather*}
T_{\alpha \beta \gamma \epsilon}(R) = 
(\delta_{\alpha \beta} \delta_{\gamma \epsilon} 
+ \delta_{\alpha \gamma} \delta_{\beta \epsilon}  +\delta_{\beta \gamma}  \delta_{\alpha \epsilon} )\,v_{41}(R) \\
+ (R_\alpha R_\beta \delta_{\gamma  \epsilon} + R_\alpha R_\gamma \delta_{\beta \epsilon} 
+ R_\alpha R_ \epsilon \delta_{\beta \gamma} + R_\beta R_\gamma \delta_{\alpha \epsilon} 
+ R_\beta R_\epsilon \delta_{\alpha \gamma}  + R_\gamma R_\epsilon \delta_{\alpha \beta})
v_{42}(R)/R^2\\
+ R_\alpha R_\beta R_\gamma R_\epsilon v_{43}(R)/R^4 ,
\end{gather*}
we explicitly find:
\begin{gather*}
T_{1111}(R)=3v_{41}(R)+6X^2v_{42}(R)/R^2+X^4v_{43}(R)/R^4\\
T_{2222}(R)=3v_{41}(R)+6Y^2v_{42}(R)/R^2+Y^4v_{43}(R)/R^4\\
T_{3333}(R)=3v_{41}(R)+6Z^2v_{42}(R)/R^2+Z^4v_{43}(R)/R^4\\
T_{1122}(R)=T_{2211}(R)=v_{41}(R)+(X^2+Y^2)v_{42}(R)/R^2+X^2Y^2v_{43}(R)/R^4\\
T_{1133}(R)=T_{3311}(R)=v_{41}(R)+(X^2+Z^2)v_{42}(R)/R^2+X^2Z^2v_{43}(R)/R^4\\
T_{2233}(R)=T_{3322}(R)=v_{41}(R)+(Y^2+Z^2)v_{42}(R)/R^2+Y^2Z^2v_{43}(R)/R^4  .\\
\end{gather*}
Now we follow Neumann and Steinhauser and Fourier transform Eq.~(\ref{fourierQuad}).  Then we assume that 
the applied field gradient is homogeneous over the entire volume (tensor $\mathsf{G}$ is a constant), the only relevant contribution of the interaction tensor is at $\mathbf{k}=0$.  Schematically, the form of the equation is:
\begin{gather}
\begin{aligned}
\frac{1}{3}
\begin{pmatrix}
\lambda\\
\mu\\
\nu\\
\end{pmatrix}
= \epsilon_0 \chi_q
\begin{pmatrix}
\lambda^\prime\\
\mu^\prime\\
\nu^\prime\\
\end{pmatrix}
+\frac{\chi_q}{24\pi}
\begin{pmatrix}
\tilde{T}_{1111}(0) & \tilde{T}_{1122}(0) & \tilde{T}_{1133}(0)\\
\tilde{T}_{2211}(0) & \tilde{T}_{2222}(0) & \tilde{T}_{2233}(0)\\
\tilde{T}_{3311}(0) & \tilde{T}_{3322}(0) & \tilde{T}_{3333}(0)\\
\end{pmatrix}
\begin{pmatrix}
\lambda\\
\mu\\
\nu\\
\end{pmatrix} .
\end{aligned}
\label{eq:schematic}
\end{gather}
The matrix in this equation is the Fourier transform of matrix $\mathsf{M}(R)$ evaluated at
$\mathbf{k}=0$, which would be written as $\tilde{M}(0)$, and shows the 
nine needed components of $\tilde{T}_{\alpha \beta \gamma \epsilon}(0)$ to be computed.
Factors such as  $X^2$, $Y^2$, and $X^2 Y^2$ get angle averaged, and the nine components
are given a temporary name.  For example,
\begin{gather*}
\tilde{T}_{1111}(0)=
\int_{\textrm{cutoff sphere}} 
\big [ 3v_{41}(R)+6X^2v_{42}(R)/R^2 + X^4\,v_{43}(R)/R^4 \big] d^3R \\ 
=12\pi \int_0^{R_c}
\big [ v_{41}(R)+\frac{2}{3} v_{42}(R) + \frac{1}{15}v_{43}(R) \big] R^2\,dR =
3B^\prime
\end{gather*}
and 
\begin{gather*}
\tilde{T}_{1122}(0)=
\int_{\textrm{cutoff sphere}} 
\big [ v_{41}(R)+(X^2+Y^2) v_{42}(R)/R^2 + X^2 Y^2\,v_{43}(R)/R^4 \big] d^3R \\
=4\pi \int_0^{R_c}
\big [ v_{41}(R)+\frac{2}{3} v_{42}(R) + \frac{1}{15}v_{43}(R) \big] R^2\,dR =
B^\prime .
\end{gather*}
Now we see that
\begin{equation*}
\mathsf{\tilde{M}}(0)=
\begin{pmatrix}
3B^\prime & B^\prime & B^\prime\\
B^\prime & 3B^\prime & B^\prime\\
B^\prime & B^\prime & 3B^\prime\\
\end{pmatrix} .
\end{equation*}
Then Eq.~(\ref{eq:schematic}) can be written as three separate equations:
\begin{gather*}
\lambda=\epsilon_0 \chi_q \lambda^\prime+\frac{\chi_q}{8\pi}(3B^\prime \lambda+B^\prime \mu +B^\prime \nu)\\
\mu=\epsilon_0 \chi_q \mu^\prime+\frac{\chi_q}{8\pi}(B^\prime \lambda+3B^\prime \mu +B^\prime \nu)\\
\nu=\epsilon_0 \chi_q \nu^\prime+\frac{\chi_q}{8\pi}(B^\prime \lambda+B^\prime \mu +3B^\prime \nu)
\end{gather*}
Using the traceless condition $\lambda+\mu+\nu=0$,
we find
\begin{gather*}
\lambda=\frac{3 \epsilon_0 \chi_q \lambda^\prime}{1-\frac{\chi_q B^\prime}{4\pi}}  \textrm{, etc.}
\end{gather*}
Comparing with Eq.~(\ref{tracedConstQuad}), 
\begin{equation}
\frac{\Theta}{3}=\mathsf{Q} - \frac{1}{3}\mathsf{1}
=\frac{\epsilon_0 \chi_q}{1-\chi_q B}\nabla \mathbf{E}^0
=\frac{\epsilon_0 \chi_q \lambda^\prime}{1-\frac{\chi_q B^\prime}{4\pi}}
\end{equation}
from which we find $B^\prime/(4\pi)=B$
and thus
\begin{equation} 
B=\int_0^{R_c}
\big [ v_{41}(R)+\frac{2}{3} v_{42}(R) + \frac{1}{15}v_{43}(R) \big] R^2\,dR .
\end{equation}

% % uncomment the following lines,
% if using chapter-wise bibliography
%
% \bibliographystyle{ndnatbib}
% \bibliography{example}