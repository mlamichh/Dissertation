%
% Modified by Sameer Vijay
% Last Change: Wed Jul 27 2005 13:00 CEST
%
%%%%%%%%%%%%%%%%%%%%%%%%%%%%%%%%%%%%%%%%%%%%%%%%%%%%%%%%%%%%%%%%%%%%%%%%
%
% Sample Notre Dame Thesis/Dissertation
% Using Donald Peterson's ndthesis classfile
%
% Written by Jeff Squyres and Don Peterson
%
% Provided by the Information Technology Committee of
%   the Graduate Student Union
%   http://www.gsu.nd.edu/
%
% Nothing in this document is serious except the format.  :-)
%
% If you have any suggestions, comments, questions, please send e-mail
% to: ndthesis@gsu.nd.edu
%
%%%%%%%%%%%%%%%%%%%%%%%%%%%%%%%%%%%%%%%%%%%%%%%%%%%%%%%%%%%%%%%%%%%%%%%%

%%%%%%%%%%%%%%%%%%%%%%%%%%%%%%%%%%%%%%%%%%%%%%%%%%%%%%%%%%%%%%%%%%%%%%%%
%
% Appendix
%
%%%%%%%%%%%%%%%%%%%%%%%%%%%%%%%%%%%%%%%%%%%%%%%%%%%%%%%%%%%%%%%%%%%%%%%%

\chapter{POINT MULTIPOLES, FIELD, AND FIELD GRADIENT }

\section{Point-multipolar interactions with a spatially-varying electric field}

We can treat objects $a$, $b$, and $c$ containing embedded collections
of charges. When we define the primitive moments, we sum over that
collections of charges using a local coordinate system within each
object.  The point charge, dipole, and quadrupole for object $a$ are
given by $C_a$, $\mathbf{D}_a$, and $\mathsf{Q}_a$, respectively.
These are the primitive multipoles which can be expressed as a
distribution of charges,
\begin{align}
C_a =&\sum_{k \, \text{in }a} q_k , \label{eq:charge} \\
D_{a\alpha} =&\sum_{k \, \text{in }a} q_k r_{k\alpha}, \label{eq:dipole}\\
Q_{a\alpha\beta} =& \frac{1}{2} \sum_{k \, \text{in }  a} q_k
r_{k\alpha}  r_{k\beta} . \label{eq:quadrupole}
\end{align}
Note that the definition of the primitive quadrupole here differs from
the standard traceless form, and contains an additional Taylor-series
based factor of $1/2$.  In Paper 1, we derived the forces and torques
each object exerts on the others.

Here we must also consider an external electric field that varies in
space: $\mathbf E(\mathbf r)$.  Each of the local charges $q_k$ in
object $a$ will then experience a slightly different field.  This
electric field can be expanded in a Taylor series around the local
origin of each object.  A different Taylor series expansion is carried
out for each object.

For a particular charge $q_k$, the electric field at that site's
position is given by:
\begin{equation}
E_\gamma + \nabla_\delta E_\gamma r_{k \delta} 
+ \frac {1}{2} \nabla_\delta \nabla_\varepsilon E_\gamma r_{k \delta}
r_{k \varepsilon} + ... 
\end{equation}
Note that the electric field is always evaluated at the origin of the
objects, and treating each object using point multipoles simplifies
this greatly.

To find the force exerted on object $a$ by the electric field, one
takes the electric field expression, and multiplies it by $q_k$, and
then sum over all charges in $a$:

\begin{align}
F_\gamma &=  \sum_{k \textrm{~in~} a} q_k \lbrace E_\gamma + \nabla_\delta E_\gamma r_{k \delta} 
+ \frac {1}{2} \nabla_\delta \nabla_\varepsilon E_\gamma r_{k \delta}
r_{k \varepsilon} + ...  \rbrace  \\
 &= C_a E_\gamma + D_{a  \delta} \nabla_\delta E_\gamma 
+ Q_{a \delta \varepsilon} \nabla_\delta \nabla_\varepsilon E_\gamma +
... 
\end{align}

Similarly, the torque exerted by the field on $a$ can be expressed as
\begin{align}
\tau_\alpha &=  \sum_{k \textrm{~in~} a} (\mathbf r_k \times q_k \mathbf E)_\alpha \\
 & =  \sum_{k \textrm{~in~} a} \epsilon_{\alpha \beta \gamma} q_k
 r_{k\beta} E_\gamma(\mathbf r_k) \\
 & = \epsilon_{\alpha \beta \gamma} D_\beta E_\gamma 
+ 2 \epsilon_{\alpha \beta \gamma} Q_{\beta \delta} \nabla_\delta
E_\gamma + ...
\end{align}

The last term is essentially identical with form derived by Torres del
Castillo and M\'{e}ndez Garrido,\cite{Torres-del-Castillo:2006uo} although their derivation
utilized a traceless form of the quadrupole that is different than the
primitive definition in use here.  We note that the Levi-Civita symbol
can be eliminated by utilizing the matrix cross product in an
identical form as in Ref. \onlinecite{Smith98}:
\begin{equation}
\left[\mathsf{A} \times \mathsf{B}\right]_\alpha = \sum_\beta
\left[\mathsf{A}_{\alpha+1,\beta} \mathsf{B}_{\alpha+2,\beta}
  -\mathsf{A}_{\alpha+2,\beta} \mathsf{B}_{\alpha+1,\beta} 
\right]
\label{eq:matrixCross}
\end{equation}
where $\alpha+1$ and $\alpha+2$ are regarded as cyclic permuations of
the matrix indices.  In table \ref{tab:UFT} we give compact
expressions for how the multipole sites interact with an external
field that has exhibits spatial variations.

\begin{table}
\caption{Potential energy $(U)$, force $(\mathbf{F})$, and torque
  $(\mathbf{\tau})$ expressions for a multipolar site embedded in an
  electric field with spatial variations, $\mathbf{E}(\mathbf{r})$.
\label{tab:UFT}}
\centering
\begin{tabular}{r|ccc}
  & Charge & Dipole & Quadrupole \\ \hline
$U$ &  $C \phi(\mathbf{r})$ & $-\mathbf{D} \cdot \mathbf{E}(\mathbf{r})$ & $- \mathsf{Q}:\nabla \mathbf{E}(\mathbf{r})$ \\
$\mathbf{F}$ & $C \mathbf{E}(\mathbf{r})$ & $+\mathbf{D} \cdot \nabla \mathbf{E}(\mathbf{r})$ &  $+\mathsf{Q} : \nabla\nabla\mathbf{E}(\mathbf{r})$ \\
$\mathbf{\tau}$ & & $\mathbf{D} \times \mathbf{E}(\mathbf{r})$ & $+2 \mathsf{Q} \times \nabla \mathbf{E}(\mathbf{r})$
\end{tabular}
\end{table}

\newpage

\section{Gradient of the field due to quadrupolar polarization}
\label{singularQuad}
In this section, we will discuss the gradient of the field produced by
quadrupolar polarization. For this purpose, we consider a distribution
of charge ${\rho}(r)$ which gives rise to an electric field
$\vec{E}(r)$ and gradient of the field $\vec{\nabla} \vec{E}(r)$
throughout space. The total gradient of the electric field over volume
due to the all charges within the sphere of radius $R$ is given by
(cf. Jackson equation 4.14):
\begin{equation}
\int_{r<R} \vec{\nabla}\vec{E}\;d^3r = -\int_{r=R} R^2 \vec{E}\;\hat{n}\; d\Omega
\label{eq:8}
\end{equation}
where $d\Omega$ is the solid angle and $\hat{n}$ is the normal vector
of the surface of the sphere which is equal to
$sin[\theta]cos[\phi]\hat{x} + sin[\theta]sin[\phi]\hat{y} +
cos[\theta]\hat{z}$
in spherical coordinates.  For the charge density ${\rho}(r')$, the
total gradient of the electric field can be written as (cf. Jackson
equation 4.16),
\begin{equation}
\int_{r<R} \vec{\nabla}\vec{E}\; d^3r=-\int_{r=R} R^2\; \vec{\nabla}\Phi\; \hat{n}\; d\Omega  =-\frac{1}{4\pi\;\epsilon_o}\int_{r=R} R^2\; \vec{\nabla}\;\left(\int \frac{\rho(r')}{|\vec{r}-\vec{r'}|}\;d^3r'\right) \hat{n}\; d\Omega
\label{eq:9}
\end{equation}
The radial function in the equation (\ref{eq:9}) can be expressed in
terms of spherical harmonics as (cf. Jackson equation 3.70),
\begin{equation}
\frac{1}{|\vec{r} - \vec{r'}|} = 4\pi \sum_{l=0}^{\infty}\sum_{m=-l}^{m=l}\frac{1}{2l+1}\;\frac{{r^l_<}}{{r^{l+1}_>}}\;{Y^*}_{lm}(\theta', \phi')\;Y_{lm}(\theta, \phi)
\label{eq:10}
\end{equation}
If the sphere completely encloses the charge density then $ r_< = r'$ and $r_> = R$. Substituting equation (\ref{eq:10}) into (\ref{eq:9}) we get,
\begin{equation}
\begin{split}
\int_{r<R} \vec{\nabla}\vec{E}\;d^3r &=-\frac{R^2}{\epsilon_o}\int_{r=R} \; \vec{\nabla}\;\left(\int \rho(r')\sum_{l=0}^{\infty}\sum_{m=-l}^{m=l}\frac{1}{2l+1}\;\frac{{r'^l}}{{R^{l+1}}}\;{Y^*}_{lm}(\theta', \phi')\;Y_{lm}(\theta, \phi)\;d^3r'\right) \hat{n}\; d\Omega \\
 &= -\frac{R^2}{\epsilon_o}\sum_{l=0}^{\infty}\sum_{m=-l}^{m=l}\frac{1}{2l+1}\;\int \rho(r')\;{r'^l}\;{Y^*}_{lm}(\theta', \phi')\left(\int_{r=R}\vec{\nabla}\left({R^{-(l+1)}}\;Y_{lm}(\theta, \phi)\right)\hat{n}\; d\Omega \right)d^3r
'
\end{split}
\label{eq:11}
\end{equation} 
The gradient of the product of radial function and spherical harmonics
is given by (cf. Arfken, p.811 eq. 16.94):
\begin{equation}
\begin{split}
\vec{\nabla}\left[ f(r)\;Y_{lm}(\theta, \phi)\right] = &-\left(\frac{l+1}{2l+1}\right)^{1/2}\; \left[\frac{\partial}{\partial r}-\frac{l}{r} \right]f(r)\; Y_{l, l+1, m}(\theta, \phi)\\ &+ \left(\frac{l}{2l+1}\right)^{1/2}\left[\frac
{\partial}{\partial r}+\frac{l}{r} \right]f(r)\; Y_{l, l-1, m}(\theta, \phi).
\end{split}
\label{eq:12}
\end{equation}
Using equation (\ref{eq:12}) we get,
\begin{equation}
\vec{\nabla}\left({R^{-(l+1)}}\;Y_{lm}(\theta, \phi)\right) = [(l+1)(2l+1)]^{1/2}\; Y_{l,l+1,m}(\theta, \phi) \; \frac{1}{R^{l+2}},
\label{eq:13}
\end{equation}
where $ Y_{l,l+1,m}(\theta, \phi)$ is the vector spherical harmonics
which can be expressed in terms of spherical harmonics as shown in
below (cf. Arfkan p.811),
\begin{equation}
Y_{l,l+1,m}(\theta, \phi) = \sum_{m_1, m_2} C(l+1,1,l|m_1,m_2,m)\; {Y_{l+1}}^{m_1}(\theta,\phi)\; \hat{e}_{m_2},
\label{eq:14}
\end{equation}
where $C(l+1,1,l|m_1,m_2,m)$ is a Clebsch-Gordan coefficient and
$\hat{e}_{m_2}$ is a spherical tensor of rank 1 which can be expressed
in terms of Cartesian coordinates,
\begin{equation}
{\hat{e}}_{+1} = - \frac{\hat{x}+i\hat{y}}{\sqrt{2}},\quad {\hat{e}}_{0} = \hat{z},\quad and \quad {\hat{e}}_{-1} = \frac{\hat{x}-i\hat{y}}{\sqrt{2}}
\label{eq:15}
\end{equation} 
The normal vector $\hat{n} $ can be expressed in terms of spherical tensor of rank 1 as shown in below,
\begin{equation}
\hat{n} = \sqrt{\frac{4\pi}{3}}\left(-{Y_1}^{-1}{\hat{e}}_1 -{Y_1}^{1}{\hat{e}}_{-1} + {Y_1}^{0}{\hat{e}}_0 \right)
\label{eq:16}
\end{equation}
The surface integral of the product of $\hat{n}$ and
${Y_{l+1}}^{m_1}(\theta, \phi)$ gives,
\begin{equation}
\begin{split}
\int \hat{n}\;{Y_{l+1}}^{m_1}\;d\Omega &= \int \sqrt{\frac{4\pi}{3}}\left(-{Y_1}^{-1}{\hat{e}}_1 -{Y_1}^{1}{\hat{e}}_{-1} + {Y_1}^{0}{\hat{e}}_0 \right)\;{Y_{l+1}}^{m_1}\; d\Omega \\
&=  \int \sqrt{\frac{4\pi}{3}}\left({{Y_1}^{1}}^* {\hat{e}}_1 +{{Y_1}^{-1}}^* {\hat{e}}_{-1} + {{Y_1}^{0}}^* {\hat{e}}_0 \right)\;{Y_{l+1}}^{m_1}\; d\Omega \\
&=   \sqrt{\frac{4\pi}{3}}\left({\delta}_{l+1, 1}\;{\delta}_{1, m_1}\;{\hat{e}}_1 + {\delta}_{l+1, 1}\;{\delta}_{-1, m_1}\;{\hat{e}}_{-1}+ {\delta}_{l+1, 1}\;{\delta}_{0, m_1} \;{\hat{e}}_0\right),
\end{split}
\label{eq:17}
\end{equation}
where ${Y_{l}}^{-m} = (-1)^m\;{{Y_{l}}^{m}}^* $ and
$ \int {{Y_{l}}^{m}}^*\;{Y_{l'}}^{m'}\;d\Omega =
\delta_{ll'}\delta_{mm'} $.
Non-vanishing values of equation \ref{eq:17} require $l = 0$,
therefore the value of $ m = 0 $. Since the values of $ m_1$ are -1,
1, and 0 then $m_2$ takes the values 1, -1, and 0, respectively
provided that $m = m_1 + m_2$.  Equation \ref{eq:11} can therefore be
modified,
\begin{equation}
\begin{split}
\int_{r<R} \vec{\nabla}\vec{E}\;d^3r = &- \sqrt{\frac{4\pi}{{3}}}\;\frac{1}{\epsilon_o}\int \rho(r')\;{Y^*}_{00}(\theta', \phi')[ C(1, 1, 0|-1,1,0)\;{\hat{e}_{-1}}{\hat{e}_{1}}\\  &+ C(1, 1, 0|-1,1,0)\;{\hat{e}_{1}}{\hat{e}_{-1}}+C(
1, 1, 0|0,0,0)\;{\hat{e}_{0}}{\hat{e}_{0}} ]\; d^3r'.
\end{split}
\label{eq:18} 
\end{equation}
After substituting ${Y^*}_{00} = \frac{1}{\sqrt{4\pi}} $ and using the
values of the Clebsch-Gorden coefficients:  $ C(1, 1, 0|-1,1,0) =
\frac{1}{\sqrt{3}}, \;   C(1, 1, 0|-1,1,0)= \frac{1}{\sqrt{3}}$ and $
C(1, 1, 0|0,0,0) = -\frac{1}{\sqrt{3}}$ in equation \ref{eq:18} we
obtain,
\begin{equation}
\begin{split}
\int_{r<R} \vec{\nabla}\vec{E}\;d^3r &= -\sqrt{\frac{4\pi}{{3}}}\;\frac{1}{\epsilon_o}\int \rho(r')\;d^3r'\left({\hat{e}_{-1}}{\hat{e}_{1}}+{\hat{e}_{1}}{\hat{e}_{-1}}-{\hat{e}_{0}}{\hat{e}_{0}}\right)\\
&= - \sqrt{\frac{4\pi}{{3}}}\;\frac{1}{\epsilon_o}\;C_{total}\;\left({\hat{e}_{-1}}{\hat{e}_{1}}+{\hat{e}_{1}}{\hat{e}_{-1}}-{\hat{e}_{0}}{\hat{e}_{0}}\right).
\end{split}
\label{eq:19} 
\end{equation}
Equation (\ref{eq:19}) gives the total gradient of the field over a
sphere due to the distribution of the charges. For quadrupolar fluids
the total charge within a sphere is zero, therefore
$ \int_{r<R} \vec{\nabla}\vec{E}\;d^3r = 0 $.  Hence the quadrupolar
polarization produces zero net gradient of the field inside the
sphere.

\newpage

\section{Applied field or field gradient}
\label{Ap:fieldOrGradient}

To satisfy the condition $ \nabla . E = 0 $, within the box of molecules we have taken electrostatic potential in the following form
\begin{equation}
\begin{split}
\phi(x, y, z) =\; &-g_o \left(\frac{1}{2}(a_1\;b_1 - \frac{cos\psi}{3})\;x^2+\frac{1}{2}(a_2\;b_2 - \frac{cos\psi}{3})\;y^2 + \frac{1}{2}(a_3\;b_3 - \frac{cos\psi}{3})\;z^2 \right. \\
& \left. + \frac{(a_1\;b_2 + a_2\;b_1)}{2} x\;y + \frac{(a_1\;b_3 + a_3\;b_1)}{2} x\;z +  \frac{(a_2\;b_3 + a_3\;b_2)}{2} y\;z \right),
\end{split}
\label{eq:appliedPotential}
\end{equation} 
where $a = (a_1, a_2, a_3)$ and $b = (b_1, b_2, b_3)$ are basis vectors  determine coefficients in x, y, and z direction. And $g_o$ and $\psi$ are overall strength of the potential and angle between basis vectors respectively. The electric field derived from the above potential is,
\[\bf{E} 
=\frac{g_o}{2} \left(\begin{array}{ccc}
2(a_1\; b_1 - \frac{cos\psi}{3})\;x \;+  (a_1\; b_2 \;+ a_2\; b_1)\;y + (a_1\; b_3 \;+ a_3\; b_1)\;z \\
 (a_2\; b_1 \;+ a_1\; b_2)\;x + 2(a_2\; b_2 \;- \frac{cos\psi}{3})\;y +  (a_2\; b_3 \;+ a_3\; b_3)\;z \\
(a_3\; b_1 \;+ a_3\; b_2)\;x +  (a_3\; b_2 \;+ a_2\; b_3)y + 2(a_3\; b_3 \;- \frac{cos\psi}{3})\;z 
\end{array} \right).\] 
The gradient of the applied field derived from the potential can be written in the following form,
\[\nabla\bf{E} 
= \frac{g_o}{2}\left(\begin{array}{ccc}
2(a_1\; b_1 - \frac{cos\psi}{3}) &  (a_1\; b_2 \;+ a_2\; b_1) & (a_1\; b_3 \;+ a_3\; b_1)\;z \\
 (a_2\; b_1 \;+ a_1\; b_2) & 2(a_2\; b_2 \;- \frac{cos\psi}{3}) & (a_2\; b_3 \;+ a_3\; b_3)\;z \\
(a_3\; b_1 \;+ a_3\; b_2) & (a_3\; b_2 \;+ a_2\; b_3) & 2(a_3\; b_3 \;- \frac{cos\psi}{3})\;z 
\end{array} \right).\] 

% % uncomment the following lines,
% if using chapter-wise bibliography
%
% \bibliographystyle{ndnatbib}
% \bibliography{example}